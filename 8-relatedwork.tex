\section{Related Work}\label{section:relatedWork}
\begin{comment}
\end{comment}

In this section, we present the existing work that is related to project
popularity and IoT. 
%In our work, we study the popularity of IoT projects that are hosted on the Hackster website. Hackster allows users to host their projects, which may be replicated by other users. When a project is interesting to other users, the number of views and, consequently, the rank of that project increase (i.e., the project becomes more popular). 

\subsection{IoT devices}

%IoT is the network of physical devices (e.g., vehicles) embedded with electronics, software, sensors, actuators, and connectivity, so that these devices are allowed to collect and exchange data.

Researchers have recently studied IoT in a wide variety of problems, \ie{}
context-aware IoT
approaches~\cite{chattopadhyay2015way,d2014data,nambi2014unified,jin2014information,perera2014context,bauman2014discovering,barnaghi2016searching,li2014qos},
fault-tolerance in IoT
services~\cite{su2014decentralized,reijers2013design,zhou2015supporting}, IoT
and cloud
computing~\cite{botta2014integration,kamilaris2011smart,rao2012cloud,fox2012architecture,zaslavsky2013sensing},
and IoT service
composition~\cite{florescu2003xl,billet2014task,hachem2011ontologies,alam2010senaas,tzortzis2016semi,de2012internet,zhang2017service,huang2015context,eisenhauer2010hydra,
	su2014decentralized,patel2015enabling}.
Regarding context-aware approaches, Chattopadhyay \etal~\cite{chattopadhyay2015way} presented an
analytical method that helps developers to write IoT application without a heavy
knowledge of signal processing and specific
domains. D'Oca and Hong~\cite{d2014data} proposed a
framework with two data mining techniques (i.e.,~clustering and associated
rules) to identify the behavior of occupants related to the opening and closing of windows. The
authors found that indoor air temperature, outdoor air temperature, and
the presence of occupants were the most important factors for the opening of windows. As 
for window closing, the indoor air temperature and outdoor air temperature
are the most important factors. Regarding fault-tolerance in IoT services, Su \etal~\cite{su2014decentralized} proposed the Strip approach, which allows the achievement of failover mechanisms upon the replacement of IoT devices. The results of their research show that failures may be recovered within seconds without the need for developers and administrators in
the process. With respect to IoT and Cloud computing
integration, Botta \etal~\cite{botta2014integration} conducted a literature
review to understand the potential applications and challenges of using IoT and
Cloud computing together (\ie~the {\em CloudIoT} paradigm). The authors
identify several open issues, such as the need for more standardization in both
IoT and Cloud computing fields. Finally, with respect to IoT composition,
Tzortzis and Spyrou~\cite{tzortzis2016semi} proposed a semi-automatic approach
that allows developers to discover, consume, and interconnect IoT services in
order to create more complex services. They evaluate their approach by
interconnecting simple IoT-enabled services.% and creating a complex application for counting people in a smart meeting room.

Different from the aforementioned work, our research focuses on understanding
the metrics that share a significant relationship with the popularity of the
IoT projects rather than investigating the approaches that can improve the IoT
technology.


\subsection{Projects popularity}

There have been several studies that investigate the factors that share a
significant relationship with the popularity of software projects on
GitHub~\cite{weber2014makes,borges2016understanding,tian2015characteristics,syer2013revisiting,cosentino2017systematic,aggarwal2014co,zhu2014patterns}.
Consentino~\etal~\cite{cosentino2017systematic} summarized the metrics that
impact the popularity of GitHub projects. Similarly to our observations regarding
IoT projects, Consentino~\etal~\cite{cosentino2017systematic} noted that
documentation and involvement of popular users contribute to the popularity of
IoT projects. In fact, the importance of documentation in the popularity of
GitHub projects has already been stressed by Aggarwal~\etal~\cite{aggarwal2014co}.
Weber et al.~\cite{weber2014makes} studied a large set of features that
characterize open source projects, including both in-code features and metadata
features.  Borges et al.~\cite{borges2016understanding} used multiple linear
regressions to study the main factors that have an association with the number
of stars of GitHub projects. These factors include the programming languages,
application domains and new features of these projects. Tian et
al.~\cite{tian2015characteristics} investigated the most important factors
regarding the ratings of free Android applications.  Also, Syer
\etal~\cite{syer2013revisiting} revisited prior empirical findings in Software
Engineering for $15$ most popular mobile apps. The authors found that the
number of core developers in mobile app projects are usually smaller than large
desktop/server applications such as the Apache HTTP server. Finally,
Zhu~\etal~\cite{zhu2014patterns} have considered the number of {\em forks} to
measure the popularity of a GitHub projects instead of number of {\em
watchers}.

Compared with the aforementioned work, our study is the first to study the
popularity of IoT projects. Our study is important because IoT projects have
different characteristics from software projects---IoT projects work with
embedded systems, which lead them to be coded on a much lower level when
compared to software projects. Additionally, the hardware used in IoT projects
may significantly vary depending on the goal of the project. These unique
characteristics of IoT projects may impact on how users perceive the importance
of IoT projects when compared to software projects. We investigate five group
of metrics, which are the {\em project page}, {\em user comments}, {\em
source-code}, {\em developer activities}, and {\em developer profile} groups of
metrics. 

