\section{Related Work}\label{section:relatedWork}

In this section, we present the existing work that is related to our research.
%In our work, we study the popularity of IoT projects that are hosted on the Hackster website. Hackster allows users to host their projects, which may be replicated by other users. When a project is interesting to other users, the number of views and, consequently, the rank of that project increase (i.e., the project becomes more popular). 
As the focus of our research is project popularity, we outline related research about IoT devices and popularity of projects that are hosted on collaborative web environments.

\subsection{IoT devices}

%IoT is the network of physical devices (e.g., vehicles) embedded with electronics, software, sensors, actuators, and connectivity, so that these devices are allowed to collect and exchange data.

Researchers have recently studied IoT in a wide variety of problems.
Researchers have investigate context-aware IoT
approaches~\cite{chattopadhyay2015way,d2014data,nambi2014unified,jin2014information,perera2014context,bauman2014discovering,barnaghi2016searching,li2014qos},
fault-tolerance in IoT
services~\cite{su2014decentralized,reijers2013design,zhou2015supporting}, IoT
and cloud
computing~\cite{botta2014integration,kamilaris2011smart,rao2012cloud,fox2012architecture,zaslavsky2013sensing},
and IoT service
composition~\cite{florescu2003xl,billet2014task,hachem2011ontologies,alam2010senaas,tzortzis2016semi,de2012internet,zhang2017service,huang2015context,eisenhauer2010hydra,
	su2014decentralized,patel2015enabling}.
Regarding context-aware approaches, Chattopadhyay \etal~\cite{chattopadhyay2015way} presented an
analytical method that helps developers to write IoT application without heavy
knowledge of signal processing and specific
domains. D'Oca and Hong~\cite{d2014data} proposed a
framework with two data mining techniques (i.e.,~clustering and associated
rules) to identify occupant behaviors of a window opening and closing. The
authors found that indoor air temperature, outdoor air temperature, and
occupancy presence are the most important factors for windows opening. As to
explain window closing, the indoor air temperature and outdoor air temperature
are the most important factors. As for fault-tolerance in IoT services, Su \etal~\cite{su2014decentralized} proposed the Strip approach, which helps to achieve failover mechanisms upon the replacement of IoT devices. The results of their research show that failures may be recovered within seconds without the need for developers and administrators in
the process. Regarding IoT and Cloud computing
integration, Botta \etal~\cite{botta2014integration} conducted a literature
review to understand the potential applications and challenges of using IoT and
Cloud computing together (\ie~the {\em CloudIoT} paradigm). The authors
identify several open issues, such as the need for more standardization in both
IoT and Cloud computing fields. Finally, with respect to IoT composition,
Tzortzis and Spyrou~\cite{tzortzis2016semi} proposed a semi-automatic approach
that allows developers to discover, consume, and interconnect IoT services in
order to create more complex services. They evaluate their approach by
interconnecting simple IoT-enables services.% and creating a complex application for counting people in a smart meeting room.

Different from the aforementioned work, our research focuses on understanding the metrics that share a significant relationship with the popularity of the IoT projects rather than investigating the approaches that can improve the IoT technology.


\subsection{Projects popularity}

There have been several studies that investigate the factors that share a significant relationship with the popularity of software projects
GitHub~\cite{weber2014makes,borges2016understanding,tian2015characteristics,syer2013revisiting}.
Weber et al.~\cite{weber2014makes} studied a large set of features that
characterize open source projects, including both in-code features and metadata
features. 
%The result revealed that in-code features are more important than author metadata features. 
Borges et al.~\cite{borges2016understanding} used multiple linear regressions
to study the main factors that have an association with the number of stars of
GitHub projects. These factors include the programming languages, application
domains and new features of these projects. Tian et
al.~\cite{tian2015characteristics} investigated the most important factors
regarding the ratings of free Android applications. Also, Syer
\etal~\cite{syer2013revisiting} revisited prior empirical findings in Software
Engineering for $15$ most popular mobile apps. The authors found that the number
of core developers in mobile app projects are usually smaller than large
desktop/server applications such as the Apache HTTP server.

Compared with the aforementioned work, our study is the first to study the popularity of IoT projects in an online community. Furthermore, we investigate
five group of metrics, which are the {\em project page}, {\em user comments},
{\em source-code}, {\em developer activities}, and {\em developer profile}
groups of metrics. 

