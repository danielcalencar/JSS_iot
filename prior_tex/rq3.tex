

\noindent \textbf{\RQthree}



\vspace{0.1cm}

\noindent\textbf{\textit{Motivation.}}
IoT project development requires expertise in various domains, including programming~\cite{gyrard2015assisting,conzon2015industrial}.
In this research question, we show the importance of source-code for achieving a higher project popularity.

\vspace{0.1cm}

\noindent\textbf{\textit{Contribution of the Source-code Group of Metrics.}} \textbf{The source-code metrics contribute significantly to the popularity of IoT    projects}. The $\chi^2$ test gives a $Pr(>\chi)<2.2E-16$. Therefore, the metrics that are related to source-code also play an important role in
explaining the popularity of IoT projects.
%The most important source-code metrics are explained below.

\vspace{0.1cm}




\noindent\textbf{\textit{Significant Metrics.}} Table~\ref{table:RQ3} shows the
list of source-code metrics along with the correlated metrics. A detailed
explanation of the source-code metrics is provided at the {\em Understand}
online documentation~\cite{understandmetrics}.

\textbf{Source-code metrics have a significant relationship with the project popularity.} As shown in Table~\ref{table:RQ3}, $16$ source-code metrics, such as the number of internal instances
and the complexity of the code, share a significant relationship with the popularity of IoT projects.

\textbf{The number of lines of code has an inverse relationship with project
	popularity.} As shown in Table~\ref{table:RQ3}, as the average lines of code
(for all nested functions or methods) increases, project popularity
decreases. Previous studies, such as Noei~\etal~\cite{noei2017study}, report an
inverse relationship between the lines of code and project popularity in the Google Play Store. Hence, developers should keep source-code
as short as possible to improve project popularity.
Larger projects in terms of lines of code might be harder to maintain that can impact the project quality and popularity~\cite{li1993object,noei2017study}.


\textbf{The number of comments has a significant relationship with project popularity.} It is already known that code comments are
important for source-code comprehension~\cite{steidl2013quality}.
Similarly, our model suggests that developers should increase the number of
lines that contain comments to achieve a better project popularity. This
observation suggests that a well-documented source-code would ease the
comprehension of a given project.

\textbf{Source-code modularity makes a significant contribution to the popularity of IoT projects.} As shown in Table~\ref{table:RQ3}, the metrics
that are related to code modularity, such as the number of methods and the number of classes, share a significant positive relationship with project popularity. Previous studies have shown the important role of modular
programming in software
engineering~\cite{fitzgerald2004critical,aberdour2007achieving,moon1986object}.
For example, Fitzgerald~\cite{fitzgerald2004critical} indicates that source-code modularity is a key factor for success in open-source software. Our
results suggest that developers should write modular code while developing
their IoT projects.

\hypobox{Source-code metrics make a significant contribution to the popularity of IoT projects. Our findings suggest that developers should write a well documented and modular source-code when developing their IoT projects.}

