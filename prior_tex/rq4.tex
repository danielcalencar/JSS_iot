
\begin{table}[t]
	\scriptsize
\centering
\setlength{\tabcolsep}{3pt}
\caption{List of the developer profile metrics. The significant metrics are marked with (an) asterisk(s). The upward arrows indicate a positive relationship with project popularity, while the downward arrows indicate otherwise.}
\label{table:RQ4}
	\begin{tabularx}{0.78\columnwidth}{>{\bfseries}lrrlcl}
	\toprule
	\multirow{2}{*}{Metric} & \textbf{Mean}  & \multirow{2}{*}{\textbf{p-value}}  &     & \multirow{2}{*}{\textbf{Effect}}     &\textbf{Correlated}\\
	&\textbf{Square}&& &&\textbf{Metrics} \\
	\midrule	\midrule
\#Projects  & 4.80E-02 & 1.72E-07 & *** & $\searrow$ &                  $-$     \\
\#Skills    & 3.34E-02 & 1.28E-05 & *** & $\nearrow$ &                 $-$      \\
\#Tools    & 1.25E-02 & 7.54E-03 & **  & $\nearrow$ &                 $-$      \\
Length of Biography  & 5.64E-03 & 7.20E-02 & *   & $\nearrow$ &                  $-$     \\
Looking For a Job & 3.62E-03 & 1.49E-01 &     & $\nearrow$ &                 $-$       \\ \bottomrule
\multicolumn{6}{l}{$p-value$ codes:  `***'$<0$, `**'$<0.001$, `*'$<0.01$, `.'$<0.05$}\\ 
\end{tabularx}
\vspace{-0.1cm}
\end{table}


\noindent \textbf{\RQfour}



\vspace{0.1cm}

\noindent\textbf{\textit{Motivation.}} 
The importance of the developers' profiles is unclear for developers. Consequently, a developer may spend too much time completing the profile, while some other developer may skip filling the profile. We show that the developer profile group of metrics does not matter in comparison with the other group of metrics, such as the project page. 

\vspace{0.1cm}

\noindent\textbf{\textit{Contribution of the Developer Profile Group of Metrics.}} \textbf{The developer profile metrics does contribute to project popularity significantly}. The $\chi^2$ test denotes a non-significant contribution of the {\em developer profile} group of metrics to project popularity with a $Pr(>\chi)=0.47$. Nonetheless, some of the metrics of this group
may still be important to increase the popularity of a project.

\vspace{0.1cm}

\noindent\textbf{\textit{Significant Metrics.}} Although the {\em developer profile} group of metrics does not significantly contribute to project popularity, some individual metrics of this group may still be useful.
%Table~\ref{table:RQ4} lists the metrics of the {\em developer profile} group.



\textbf{The number of listed skills and tools of a developer have a significant relationship with project popularity.}
Table~\ref{table:RQ4} shows that as the number of listed skills and tools increase, the popularity of the respective projects increase. Our results suggest that developers should carefully provide a list of skills and tools on their profiles. However, the number of projects that a developer publishes share an inverse relationship with the popularity of projects. It seems that developers who focus on one project (or few projects) can achieve a higher popularity. 
%This observation can be due to the fact that publishing
%multiple projects requires additional management skills besides software
%development skills~\cite{preece1994taxonomy,basili1986experimentation}.

\textbf{The biography of developers share a significant relationship with project popularity.} Our model suggests that a developer whose biography is lengthier (in terms of the number of words and sentences) obtains a higher project popularity.
%Our results suggest that developers should provide a proper biography of themselves on their profiles should they intend to increase the popularity of their projects.

\hypobox{We find that the developer profile group of metrics does not make a significant contribution to project popularity. Nonetheless, our observations suggest that developers may still benefit from providing a list of skills, tools, and a biography of themselves on their profiles.}

