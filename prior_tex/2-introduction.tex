\section{Introduction}\label{introduction}
\begin{comment}
\end{comment}

The Internet of Things (IoT) is transforming human lives at an unprecedented
rate. Users can perform diverse tasks benefiting from the IoT technology. For
instance, users can remotely control the house temperature using IoT
applications. In 2017, 15\% of American residents owned a home automation
device compared to 10\% in April 2016~\cite{npd}. Additionally, according to the
International Data Corporation (IDC)~\cite{idc} there will be a market of
$\$1.7$ trillion with $30$ billion devices connected throughout the IoT
ecosystem by $2020$.  However, the heterogeneity and immature standardization
of IoT systems increase the complexity of developing IoT
systems~\cite{li2015internet,botta2014integration}. Compared to desktop/server
or mobile systems, IoT systems involve different types of devices, data
exchange protocols, and deployment environments. In addition, developing IoT
projects requires knowledge of different hardware platforms, such as
\textit{Arduino}~\cite{arduino} and \textit{Raspberry Pi}~\cite{raspberrypi},
as well as knowledge of other specific domains, such as signal
processing~\cite{chattopadhyay2015way}. 

Recently, online IoT communities have become popular among IoT developers as an
increasing number of IoT developers publish their projects online. The online
communities allow developers to learn more from each other while users also
benefit from newest IoT projects~\cite{singh2017create}. As online IoT
communities grow and serve a broader audience, they allow project developers to
create business opportunities, especially for those developers whose projects
are more popular among users. For example, the page of a popular project might
serve as an advertiser for a company that supports that project.
Also, the owners of popular projects receive {\em reputation
points}~\cite{reputation} which can be exchanged with products in online stores~\cite{store}.

Several studies in the recent years have investigated the metrics that share a
significant relationship with the popularity of software projects in online
communities~\cite{chen2011predicting, tian2015characteristics,
krutz2016examining, stewart2002exploratory}.  For example,
Tian~\etal~\cite{tian2015characteristics} studied the metrics that are
significantly associated with highly rated free Android applications.
Tian~\etal~found that the number of promotional images displayed by an app on
the Google Play Store and the size of the app are strongly correlated with
highly rated Android apps. Despite the advances that have been made by studying the popularity of software
projects in online communities, little is known about the characteristics of
the popular IoT projects.

the growing importance of IoT applications

In this paper, we conduct an exploratory study to investigate the popularity of
IoT projects in online communities.  We perform a case study of $1,232$ IoT
projects that are hosted on \textit{Hackster}~\cite{hackster}, a large online
IoT community that is sponsored by Microsoft, Intel, Google, and Amazon.  In
addition, several IoT projects on the Hackster are developed and deployed using
IoT cloud services, such as Amazon Web Service IoT Platform (AWS
IoT)~\cite{awsamazon} and Microsoft Azure~\cite{azure}.  Such projects
comprehend a wide diversity of IoT projects, such as home automation (e.g.,
remotely controlling the surveillance cameras and air conditioners in the
house).  All of the collected IoT projects in our study are based on the two
most popular IoT platforms (i.e., \textit{Arduino} and \textit{Raspberry
Pi})~\cite{hackster-survey-2016}. 

We study five groups of metrics where each group covers a unique aspect of an
IoT project, including (i) \textit{project page} that concerns textual and
graphical explanations, (ii) \textit{source-code} that examines the source-code
metrics, (iii) \textit{user comments} group that covers the comments that are
posted for each project, (iv) \textit{developer profile} that covers the
profile contents of each developer, and (v) \textit{developer activities}
that reflects each developers' activities on the Hackster. In total, we measure
$112$ metrics to explain project popularity.  First, we investigate the
popularity of IoT projects by building a linear regression model, namely a
general model, which obtains a goodness-of-fit of $0.67$ that can greatly
explain the relationship between the metrics and project
popularity~\cite{pearson1893contributions}.  To determine the contribution of
each group of metrics to project popularity, we repeat the following steps for
each group of metrics: (i) we remove all the metrics of a given group and build
another model, and (ii) we investigate the contribution of each group of
metrics by comparing the new model with the general model using the $\chi^2$
test~\cite{rice1989analyzing}.  Moreover, we identify the metrics that share a
significant relationship with project popularity.  Developers can improve the
popularity of their projects with respect to our findings.  We are in favor of
addressing the following five research questions:

\vspace{0.1cm}

\noindent\textit{\RQone}


We observe that project page metrics are significantly associated with project
popularity. Ten project page metrics, such as tags and project descriptions,
share a significant relationship with project popularity. Therefore, IoT
developers should carefully build the pages of their projects.

\vspace{0.1cm}

\noindent\textit{\RQtwo}

We find that user comments metrics contribute to project popularity
significantly.  For example, developers should reply to the questions that are
asked in the comments in order to achieve a higher project popularity.

\vspace{0.1cm}

\noindent\textit{\RQthree}

Source-code metrics, such as the number of comments in the source-code, have a
significant relationship with project popularity. Developers should make sure
that the source-code is comprehensible and reusable when publishing a project.

\vspace{0.1cm}

\noindent\textit{\RQfour}

We observed that developers' profiles do not contribute to the popularity of the associated projects significantly.

\vspace{0.1cm}

\noindent\textit{\RQfive}

Developers' activities, such as commenting on other users' posts and giving respects (i.e., thumbs-ups) to other projects, can significantly improve the popularity of the associated projects. Hackster provides a specific page for each developer that lists the complete history of a given developer's activities.

\vspace{0.1cm}

%
%The main contributions of this paper are:
%
%\begin{itemize}
%    \item We are the first to conduct a large-scale explanatory study on the popularity of IoT projects in online communities.
%    We study $112$ metrics of IoT projects from five different groups of metrics, such as the source-code and developer activities.
%    \item We apply statistical analyses on our proposed metrics and discover $31$ metrics that share a significant relationship with the popularity of IoT projects. Moreover, we provide actionable suggestions to help IoT developers increase the popularity of their projects in online communities.
%\end{itemize}


\textit{Paper organization.}
In Section~\ref{datacol}, we explain the data collection process. In Section~\ref{sec:measure}, we describe the metrics that we use to explain project popularity.
We describe our approach in Section~\ref{section:approach}.
In Section~\ref{section:results}, we report and discuss the results of the experiments.
In Section~\ref{section:threats}, we discuss the potential threats to the validity of our study. We outline the related research in Section~\ref{section:relatedWork}. Finally, we draw conclusions in Section~\ref{section:conclusion}.

