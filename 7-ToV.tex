\section{Threats to Validity}\label{section:threats}
\begin{comment}
\end{comment}

In this section, we discuss the potential threats to the validity of our work.

\textbf{Threats to conclusion validity} concern if theoretical constructs are
measured and interpreted correctly~\cite{Shull:2007:GAE:1324786}.  For each
observation, we perform conjectures as tentative explanations for what we
observe. For example, although we study the questions that are asked by users
and developers, such questions may not always express
uncertainty~\cite{ebert2017confusion}. We based ourselves on the related
research whenever possible.  Nonetheless, We do not claim that such conjectures
are true behind our observations. Future work could deeply investigate our
observations and improve the body of knowledge about the popularity of IoT
projects. Furthermore, some metrics may not be actionable for project
developers, such as the number of comments. We kept such metrics in our study
as control metrics~\cite{noru2012ibm}.


\textbf{Threats to internal validity} concern our analysis methods and
selection of subject systems~\cite{Shull:2007:GAE:1324786}.  We study $112$
metrics that are collected from the Hackster online community and the GitHub
repositories to explain project popularity. However, we do not claim a complete
coverage of all the explanatory metrics that could explain project popularity.
For example, developers' response time can be another explanatory metric.
However, the Hackster presents the response dates with varying granularities,
including day, weak, month, and year; as the dates get older the granularity
increases. For example, $24\%$ of our collected comments are dated as
\textit{``a year ago''}. Therefore, we could not include the response time on
our study.  Future work may study more metrics to explain project popularity
better. 

\textbf{Threats to external validity} concern the possibility to generalize the
findings~\cite{Shull:2007:GAE:1324786}. Our goal is to understand the
popularity of IoT projects in this paper. However, our results cannot be
extended for developers and entrepreneurs who look for something other than
project popularity, such as the monetary benefits. Nonetheless, project
popularity brings more users that can help a project to sell more in the
future.

