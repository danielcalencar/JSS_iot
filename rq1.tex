\section{Results}\label{section:results}

In this section, we present our motivations and the results of each research question. Each research
question $Q$ analyzes a specific group $\partial_Q$ of metrics. First, we examine how significant is the relationship that $\partial_Q$ shares with the popularity of IoT projects.
Then, we highlight the most significant metrics of $\partial_Q$ as indicated by the $model_{general}$.



\noindent\textbf{\RQone}



\vspace{0.1cm}

\noindent\textbf{\textit{Motivation.}}
As mentioned in Section~\ref{datacol}, project pages provide descriptions and details of a given project. However, the importance of the project pages is not clear for developers. For example, a developer may underestimate the relevance of project estimated time (i.e., the time needed to complete a project) to project popularity. In this research question, we show that the project page group of metrics shares a significant relationship with project popularity.



\vspace{0.1cm}


\begin{table}[t]
\centering
\scriptsize
\caption{List of the project page metrics. The significant metrics are marked with (an) asterisk(s). The upward arrows indicate a positive relationship with project popularity, while the downward arrows indicate otherwise.}
	\setlength{\tabcolsep}{3.8pt}
\label{table:RQ1}
	\begin{tabularx}{0.8\columnwidth}{>{\bfseries}lrrlcl}
	\toprule
\multirow{2}{*}{Metric} & \textbf{Mean}  & \multirow{2}{*}{\textbf{p-value}}  &     & \multirow{2}{*}{\textbf{Effect}}     &\textbf{Correlated}\\
&\textbf{Square}&& &&\textbf{Metrics} \\
	\midrule	\midrule
\#Tags                 & 4.56E-01 & 1.79E-54 & *** & $\nearrow$ & $-$                  \\
License                & 1.90E-02 & 2.09E-28 & *** & $-$        & $-$                  \\
Difficulty             & 5.87E-02 & 5.74E-27 & *** & $-$        & $-$                  \\
Platform               & 1.13E-01 & 1.41E-15 & *** & $-$ & $-$                  \\
Estimated Time        & 9.58E-02 & 1.93E-13 & *** & $\searrow$ & $-$                  \\
Type                   &  9.58E-02 & 1.94E-13 & *** & $-$        & $-$                  \\
\#Story                & 3.02E-02 & 3.26E-05 & *** & $\searrow$ & $-$                  \\
Description Length    & 1.63E-02 & 2.23E-03 & **  & $\nearrow$ & $-$                  \\
Schematics & 1.06E-02 & 1.36E-02 & *   & $\searrow$ & $-$                  \\
\#Software Tools             & 9.64E-03 & 1.87E-02 & *   & $\nearrow$ & $-$                  \\
\#Hardware Tools            & 4.82E-03 & 9.63E-02 & .   & $\nearrow$ & $-$                  \\
\#Links                & 3.33E-03 & 1.67E-01 &     & $\nearrow$ & $-$                  \\
YouTube           & 2.30E-04 & 7.15E-01 &     & $\nearrow$ & $-$              \\ \bottomrule
\multicolumn{6}{l}{$p-value$ codes:  `***'$<0$, `**'$<0.001$, `*'$<0.01$, `.'$<0.05$}\\ 
\end{tabularx}
%\vspace{-0.1cm}
\end{table}


\begin{table}[t]
	\scriptsize
	\centering
	\caption{List of all the licenses along with the effect of each license on project popularity. The significant metrics are marked with (an) asterisk(s). The upward arrows indicate a positive relationship with project popularity, while the downward arrows indicate otherwise.}
	\setlength{\tabcolsep}{2pt}
	\label{table:license}
	\begin{tabularx}{0.8\columnwidth}{>{\bfseries}lrlc||>{\bfseries}lrlc}
		\toprule
		\multirow{2}{*}{License}      & \multirow{2}{*}{\textbf{p-value}}  &    & \multirow{2}{*}{\textbf{Effect}}     & \multirow{2}{*}{License}  & \multirow{2}{*}{\textbf{p-value}}  &    & \multirow{2}{*}{\textbf{Effect}}     \\
		&&&&&&&\\ \midrule\midrule
		Beerware     & 0.54 &    & $\nearrow$ & CC0      & 0.59 &    & $\nearrow$ \\
		BSD-3-Clause & 0.87 &    & $\searrow$ & CERN-OHL & 0.76 &    & $\nearrow$ \\
		CC BY        & 0.40 &    & $\nearrow$ & GPL      & 0.42 &    & $\nearrow$ \\
		CC BY-NC     & 0.00 & ** & $\nearrow$ & GPL3+    & 0.86 &    & $\searrow$ \\
		CC BY-NC-ND  & 0.44 &    & $\nearrow$ & LGPL     & 0.78 &    & $\searrow$ \\
		CC BY-NC-SA  & 0.88 &    & $\searrow$ & MIT      & 0.00 & ** & $\nearrow$ \\
		CC BY-ND     & 0.80 &    & $\searrow$ & MPL-2.0  & 0.86 &    & $\nearrow$ \\
		CC BY-SA     & 0.50   &    & $\nearrow$ &TAPR-OHL          &  0.86        &    & $\nearrow$          \\ \bottomrule
		\multicolumn{8}{l}{$p-value$ codes:  `***'$<0$, `**'$<0.001$, `*'$<0.01$, `.'$<0.05$}\\ 
	\end{tabularx}
\vspace{-0.1cm}
\end{table}
 

%\begin{table}[t]
\centering
\scriptsize
\caption{List of all the difficulties along with the effect of each license on the popularity of the projects. The significant metrics are marked with (an) asterisk(s).}
\label{table:difficulty}
\begin{tabular}{|>{\bfseries}l|rl|c|}
	\hline
\multirow{2}{*}{Difficulty}      & \multirow{2}{*}{\textbf{p-value}}  &     & \multirow{2}{*}{\textbf{Effect}}     \\
&&&\\ \hline\hline
Easy         & 3.89E-06 & *** & $\nearrow$ \\
Intermediate & 9.44E-07 & *** & $\nearrow$ \\
Advanced     & 6.32E-07 & *** & $\nearrow$ \\
Super Hard   & 0.20  &     & $\nearrow$ \\ \hline
\multicolumn{4}{l}{$p-value$ codes:  `***'$<0$, `**'$<0.001$, `*'$<0.01$, `.'$<0.05$}\\ 
\end{tabular}
%\vspace{-0.2cm}
\end{table}
 

\noindent\textbf{\textit{Contribution of the Project Page Group of Metrics.}} \textbf{The project page metrics make a significant contribution to the popularity of IoT projects}. The $\chi^2$ test indicates a $Pr(>\chi)=9.56E-15$.
Therefore, the project page group of metrics is significantly related to project popularity. To achieve a higher popularity, developers should pay special attention to the project pages rather than just focusing on one aspect of a project, such as source-code.

\vspace{0.1cm}

\noindent\textbf{\textit{Significant Metrics.}}
Table~\ref{table:RQ1} shows the list of the {\em project page} metrics. The first column shows the mean square of each metric. The mean square is the variances for the {\em degrees of freedom} that are used to estimate the
variance~\cite{fisher1992statistical}. The degrees of freedom of {\em difficulty} is $3$, {\em license} is $16$, and {\em type} is $3$ as they are factor metrics that hold more than $2$ different values (e.g.,~see Table~\ref{table:license}). The rest of the metrics have one degree of freedom. The second column of Table~\ref{table:RQ1} shows the associated $p-value$ to the $F$-test~\cite{lomax2013statistical} (i.e., $Pr(>F)$).
%The significant metrics are marked with (an) asterisk(s).
The third column shows the effect of each metric where the upward arrows indicate a positive relationship with project popularity, while the downward arrows indicate otherwise.
The correlated metrics are listed in the last column.

\input{Tables/tags.tex}

\begin{table*}[t]
	\centering
	\caption{List of the user comments metrics. The significant metrics are marked with (an) asterisk(s). The upward arrows indicate a positive relationship with project popularity, while the downward arrows indicate otherwise.}
	\label{table:RQ2}
\begin{tabular}{|>{\bfseries}l|r|rl|c|l|}
	\hline
	\multirow{2}{*}{Metric} & \textbf{Mean}  & \multirow{2}{*}{\textbf{p-value}}  &     & \multirow{2}{*}{\textbf{Effect}}     &\textbf{Correlated}\\
	&\textbf{Square}&& &&\textbf{Metrics} \\
	\hline	\hline
		\#Comments with positive sentiment scores       & 1.85E+00 & 2.35E-176 & *** & $\nearrow$ & Ratio of comments with positive sentiment scores                                                                                                   \\
		\#Comments with negative sentiment scores       & 1.13E+00 & 6.28E-119 & *** & $\searrow$ & \#Comments with neutral sentiment scores                                                                                                           \\
		\#Questions addressed by developers             & 5.85E-01 & 6.12E-68  & *** & $\nearrow$ & \begin{tabular}[c]{@{}l@{}}Ratio of questions addressed by developers,\\ \#Addressed questions, Ratio of addressed questions\end{tabular}                                                    \\
		\#Comments                                      & 7.11E-02 & 2.18E-10  & *** & $\nearrow$ & \#Questions                                                                                                                                        \\
		Ratio of questions                              & 8.19E-03 & 3.02E-02  & *   & $\searrow$ &        $-$                                                                                                                                             \\
		\begin{tabular}[c]{@{}l@{}}Ratio of comments with negative sentiment\\scores\end{tabular} & 7.37E-03 & 3.98E-02  & *   & $\searrow$ & Ratio of comments with neutral sentiment scores                                                                                                    \\
		\#Developers' replies                           & 3.70E-03 & 1.45E-01  &     & $\searrow$ & \begin{tabular}[c]{@{}l@{}}Ratio of reply/comment, \#Replies, Discussions\\ (Maximum and Average), Ratio of developers'\\replies/comments,  Ratio of developers' replies/replies\end{tabular}   \\ \hline
		\multicolumn{6}{l}{$p-value$ codes:  `***'$<0$, `**'$<0.001$, `*'$<0.01$, `.'$<0.05$}\\                                                                                                                                   
	\end{tabular}
\vspace{-0.1cm}
\end{table*}

\textbf{The number of tags has the most significant relationship with project popularity in comparison with other project page metrics.} As the number of tags increases, the popularity of a project is subject to increase.
This observation might be related to the vital importance of the tags as an effective mechanism to attract users~\cite{zarrella2009social}.
%Therefore, developers should assign tags to their projects properly.

\textbf{The project license has a significant relationship with the popularity of projects}.
As shown in Table~\ref{table:RQ1}, project license make a significant contribution to the model that explains project popularity with a $p-value=1.90E-02$.
%Table~\ref{table:license} shows a list of all licenses along with the $p-value$ and effect of each license.
As shown in Table~\ref{table:license}, \textit{CC BY} and \textit{MIT} share a positive significant relationship with project popularity. Similar to our observations, Sen~\cite{sen2006open} found that there exists a significant relationship between license types and FLOSS projects~\cite{crowston2004towards} popularity.
Consequently, developers should keep all the restrictions of the available licensing mechanisms in mind before licensing their projects.


\textbf{Setting difficulties does not impact project popularity negatively.}
Although {\em difficulty} shares a significant relationship with project popularity (see Table~\ref{table:RQ1}), the value of difficulty does not negatively impact project popularity. However, setting the difficulty to \textit{super hard} is the only case that does not contribute to a higher project popularity significantly.
Therefore, it is better for developers to avoid setting their projects as {\em super~hard} projects.

\textbf{A shorter estimated time is associated with a higher project popularity.} The estimated time of a project shares a significant inverse relationship with project popularity. Developers should keep the estimated time as short as possible since projects with a shorter estimated time tend to be more popular.


\textbf{The number of software tools shares a significant relationship with project popularity, while the number of hardware tools does not.} As shown in Table~\ref{table:RQ1}, the number of software tools that are used to build a project has a significant positive relationship with project popularity. On the other hand, the number of used hardware tools does not have a significant contribution to project popularity. The significant impact of the {\em number of software tools} metric may be explained by users being more attracted to projects that use software tools which with such users are more familiar.

\textbf{Project explanations contribute significantly to project popularity.}
As shown in Table~\ref{table:RQ1}, the number of stories, description length, and availability of schematics all play an important role in the popularity of IoT projects.
Although developers should keep the number of stories as low as possible, developers should provide more content for descriptions.
Providing schematics has an inverse relationship with the popularity. By investigating our projects more deeply, we observe that harder projects tend to provide more schematics.
For example, $85\%$ of the advanced projects provide schematics.
Moreover, projects that are marked as \textit{in progress} share a significant inverse relationship with project popularity.
Developers should publish rich and simple instructions for their final work in order to achieve higher popularity.


\begin{sidewaystable}
	\scriptsize
	\tabcolsep=0.1cm
\centering
\caption{List of the source-code metrics. The significant metrics are marked with (an) asterisk(s). The upward arrows indicate a positive relationship with project popularity, while the downward arrows indicate otherwise.}
\label{table:RQ3}
	\begin{tabularx}{\columnwidth}{>{\bfseries}lrrlcl}
	\toprule
	\multirow{2}{*}{Metric} & \textbf{Mean}  & \multirow{2}{*}{\textbf{p-value}}  &     & \multirow{2}{*}{\textbf{Effect}}     &\textbf{Correlated}\\
	&\textbf{Square}&& &&\textbf{Metrics} \\
	\midrule	\midrule
CountDeclInstanceVariableInternal          & 1.21E-01 & 1.90E-16 & *** & $\nearrow$ &             $-$                                                                                                                                                                                                                                                                                                                                                     \\
AltAvgLineCode                             & 1.22E-01 & 1.38E-16 & *** & $\searrow$ & AltAvgLineBlank, AltAvgLineComment                                                                                                                                                                                                                                                                                                                              \\
CountDeclMethodDefault                     & 3.71E-02 & 4.27E-06 & *** & $\nearrow$ &               $-$                                                                                                                                                                                                                                                                                                                                                   \\
CountLineComment                           & 3.21E-02 & 1.85E-05 & *** & $\nearrow$ & AvgLineComment                                                                                                                                                                                                                                                                                                                                                  \\
CountDeclClassMethod                       & 2.43E-02 & 1.93E-04 & *** & $\nearrow$ &             $-$                                                                                                                                                                                                                                                                                                                                                     \\
CountDeclMethodInternal                    & 2.35E-02 & 2.45E-04 & *** & $\nearrow$ & CountDeclProperty, CountDeclPropertyAuto                                                                                                                                                                                                                                                                                                                        \\
CountDeclMethodProtectedInternal           & 1.49E-02 & 3.52E-03 & **  & $\searrow$ &         $-$                                                                                                                                                                                                                                                                                                                                                         \\
AvgEssential                               & 1.48E-02 & 3.64E-03 & **  & $\nearrow$ &   $-$                                                                                                                                                                                                                                                                                                                                                               \\
CountDeclInstanceVariableProtected         & 1.36E-02 & 5.26E-03 & **  & $\nearrow$ &   $-$                                                                                                                                                                                                                                                                                                                                                               \\
CountPath                                  & 1.03E-02 & 1.50E-02 & *   & $\nearrow$ & CountPathLog                                                                                                                                                                                                                                                                                                                                                    \\
CountLinePreprocessor                      & 6.26E-03 & 5.80E-02 & .   & $\searrow$ &  $-$                                                                                                                                                                                                                                                                                                                                                                \\
CountDeclInstanceVariablePrivate           & 5.35E-03 & 7.98E-02 & .   & $\searrow$ &     $-$                                                                                                                                                                                                                                                                                                                                                             \\
CountClassDerived                          & 5.34E-03 & 8.00E-02 & .   & $\searrow$ &   $-$                                                                                                                                                                                                                                                                                                                                                               \\
\begin{tabular}[c]{@{}l@{}}CountDeclInstanceVariableProtected-\\Internal\end{tabular} & 5.75E-03 & 6.93E-02 & .   & $\searrow$ &    $-$                                                                                                                                                                                                                                                                                                                                                              \\
CountDeclMethodPublic                      & 3.36E-03 & 1.65E-01 &     & $\nearrow$ & \begin{tabular}[c]{@{}l@{}}CountClassBase, CountClassCoupled, CountDeclClass,\\ CountDeclInstanceMethod, CountDeclInstanceVariable, \\CountDeclMethod, CountDeclMethodAll, CountDeclMethodPrivate \end{tabular}                                                                                                                                                                                             \\
CountDeclInstanceVariablePublic            & 2.30E-03 & 2.50E-01 &     & $\searrow$ &  $-$                                                                                                                                                                                                                                                                                                                                                                \\
CountDeclClassVariable                     & 2.46E-03 & 2.35E-01 &     & $\nearrow$ & CountDeclMethodProtected                                                                                                                                                                                                                                                                                                                                        \\
CountDeclMethodFriend                      & 1.39E-03 & 3.72E-01 &     & $\nearrow$ &     $-$                                                                                                                                                                                                                                                                                                                                                             \\
CountLineInactive                          & 1.07E-03 & 4.33E-01 &     & $\nearrow$ &    $-$                                                                                                                                                                                                                                                                                                                                                              \\
AltCountLineCode                           & 9.10E-04 & 4.69E-01 &     & $\nearrow$ & AltCountLineBlank, AltCountLineComment                                                                                                                                                                                                                                                                                                                          \\
Knots                                      & 5.40E-04 & 5.78E-01 &     & $\nearrow$ & CountInput, CountOutput, CountSemicolon                                                                                                                                                                                                                                                                                                                         \\
CountDeclFile                              & 4.50E-04 & 6.11E-01 &     & $\searrow$ &    $-$                                                                                                                                                                                                                                                                                                                                                              \\
CountDeclMethodConst                       & 2.60E-04 & 6.99E-01 &     & $\searrow$ &    $-$                                                                                                                                                                                                                                                                                                                                                              \\
CountStmtEmpty                             & 1.60E-04 & 7.64E-01 &     & $\nearrow$ &   $-$                                                                                                                                                                                                                                                                                                                                                               \\
AvgLineBlank                               & 1.20E-04 & 7.89E-01 &     & $\nearrow$ &  $-$                                                                                                                                                                                                                                                                                                                                                                \\
CountLine                                  & 6.00E-05 & 8.53E-01 &     & $\searrow$ & \begin{tabular}[c]{@{}l@{}}AvgCyclomatic, AvgCyclomaticModified, AvgCyclomaticStrict,\\ AvgLine, AvgLineCode, CountDeclFunction, CountLine,\\ CountLineBlank, CountLineCode, CountLineCodeDecl,\\ CountLineCodeExe, CountStmt, CountStmtDecl, CountStmtExe,\\ Cyclomatic, CyclomaticModified, CyclomaticStrict, Essential\end{tabular} \\ \bottomrule
\multicolumn{6}{l}{$p-value$ codes:  `***'$<0$, `**'$<0.001$, `*'$<0.01$, `.'$<0.05$}\\ 
\end{tabularx}
\vspace{-0.1cm}
\end{sidewaystable}


\noindent\textbf{\textit{Discussion.}}
Due to the importance of the number of tags and the number of software tools (see Table~\ref{table:RQ1}), we perform deeper analyses of these metrics by adding all the distinct tags (i.e., $1,646$ tags) and all the distinct software tools (i.e., $518$ software tools) of the subject projects to our model as factor metrics.
By knowing the exact tags and software tools that can improve project popularity, developers should utilize such tags and software tools to improve the popularity of their projects.
%We extract all the distinct tags and all the distinct software tools from our subject projects.
We identify $65$ tags that have a significant relationship with the popularity of the projects. A complete list of all important tags are listed in Table~\ref{table:tags}. We also found three software tools that significantly contribute to the popularity of the projects, including \emph{Microsoft Windows IoT Core}, \emph{Here Maps API}, and \emph{Google Earth}.

\hypobox{The project page metrics make a significant contribution to project popularity. Our results suggest that developers should carefully tag their projects and should provide enough didactic material for their projects.}

