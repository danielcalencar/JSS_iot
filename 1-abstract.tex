\begin{comment}
\end{comment}
Online Internet of Things (IoT) communities, such as the Hackster, allow IoT
developers to publish their projects for a wider audience of users. A popular IoT project
attracts more prospective users that opens more business opportunities. Despite
the growing adoption of IoT technologies in business, little is known about the
characteristics of popular IoT projects. The goal of our study is to explain
the main characteristics of popular IoT projects. We study 1,232 IoT projects
that are hosted on the Hackster---a large online IoT community. We analyze five
groups of metrics that share distinct aspects of a project, including (i)
project page, (ii) source-code, (iii) user comments, (iv) developer profile,
and (v) developer activities. We explain the popularity of the subject projects
by drawing a linear regression model with a goodness-of-fit of 0.67 (i.e., a
general model). Using our general model, we identify the metrics that have a
significant relationship with project popularity, such as the project tags and
comment follow-ups. To highlight the contribution of each group of metrics, we
remove all of the metrics of a given group and build another model. Then, we
compare the new model with the general model using the $\chi^2$ test. We repeat
the process above for each group of metrics. We find that the project page,
source-code, user comments, and developer activities make significant
contributions to the popularity of IoT projects. Having considered our
findings, developers would be able to tackle business opportunities by
understanding the characteristics that can improve the popularity of IoT
projects.
